\documentclass[UTF8]{ctexart}

\title{生物统计课程论文}
\author{徐伟泽\\2018302110174\\预防兽医学,动物医学院\\华中农业大学}

\usepackage[%
backend=biber,
autocite=superscript,
sorting=none,
style=numeric,
]{biblatex}
\addbibresource{ref.bib}

\usepackage{graphicx}
\usepackage{geometry}
 \geometry{
 a4paper,
 total={170mm,257mm},
 left=20mm,
 top=20mm,
 }

\begin{document}
  \maketitle

  \textbf{摘要}:
  本文探讨了几种机器学习模型在DNA序列分类与回归问题
  上的应用,结合具体的数据集对几种模型的相应问题上的表现进行了
  评估。其中分类问题数据来自于 DAP-Seq 数据得到的
  转录因子结合位点序列数据,回归数据来自于 CRISPR-Cas9 敲除实验
  对癌细胞生长影响数据。通过比较各种方法应用在相应数据上得到的结果,
  讨论了机器学习模型在核酸序列数据的应用方法以及存在的问题。

  \section{前言}
  目前机器学习、统计学习方法已经被广泛应用于处理DNA序列与基因组学相关实验数据
  的分类与回归问题。

  \section{材料与方法}
  \subsection{数据}
  \subsubsection{分类问题}
  分类问题数据来自于2016年发表于 Nature Biotechnology 的 DAP-Seq 数据\autocite{o2016cistrome}。
  数据包括正样本与负样本共5834条长度为201bp的 DNA 序列。
  其中正样本为实验得出的TF(Transcription Factor,译:转录因子)结合位点附近的DNA序列,负样本为
  染色体上随机抽取的相等长度的序列。

  \subsubsection{回归问题}
  回归问题的数据集来自于CRISPR-Cas9 敲除 p53 enhancer 筛选实验\autocite{korkmaz2016functional}。
  预测的根据为敲除位点附近的核酸序列,预测目标为敲除后癌细胞生长的 Enrichment Z-Score。

  \subsection{序列特征提取}
  \subsubsection{k-mer计数}
  k-mer 计数是一种常用的较为朴素的序列特征提取方法。k-mer 指的是
  序列中长度为 k 的子序列,当序列为DNA时,所有可能的 k-mer 种类
  数量为 $4^k$,所以k-mer 计数特征可表示为一个长度为 $4^k$ 的
  向量 $F_k(s) = [c_1, ... c_i, ... c_{4_k}] $ 其中 $c_i$ 为第 $i$
  个 k-mer 再序列 $s$ 中出现的次数。 对于 k-mer 计数特征,k 是唯一的参数,
  在之前的研究中一般将 k 设置为 6 左右\autocite{ghandi2014enhanced,zeng2018prediction}。

  \subsubsection{k-mer sentence 与 Seq2Vec}
  除了将 k-mer 计数作为序列特征,之前的研究中还有研究者借鉴自然语言
  处理中的方法,将序列视为由 k-mer 作为词的句子。然后将句子嵌入到
  欧式空间中,将嵌入后得到的向量作为特征\autocite{zeng2018prediction}。

  \subsubsection{Recurrent Neural Network}
  既然将DNA序列特征提取能够类比于自然语言序列的特征提取,那么可以进一步
  的借鉴自然语言处理中的其他技术来进行DNA序列特征提取。
  比如使用 RNN (Recurrent Neural Network, 译:递归神经网络)
  或者 LSTM (Long Short-Term Memory)\autocite{hochreiter1997long}、
  GRU(Gated Recurrent Unit)\autocite{cho2014learning}
  等基于神经网络的技术来做特征提取。当然之前已经有研究者使用
  这类技术提取特征用于 TF 结合位点预测\autocite{shen2018recurrent}。

  \subsection{分类模型}
  \subsubsection{Support Vector Machine}
  \subsubsection{Random Forest}

  \subsection{回归模型}
  \subsubsection{Linear Regression}
  \subsubsection{Lasso 与 Ridge regression}
  \subsubsection{Gradient Boosting Regression Tree}

  \subsection{结果评价}
  \subsubsection{分类结果评价}
  \subsubsection{回归结果评价}
  \subsubsection{交叉检验}

  \section{结果}
  \subsection{分类问题}
  \subsection{回归问题}

  \section{讨论}
 
  \printbibliography

\end{document}